\documentclass[final,5p,times,twocolumn]{elsarticle}
\usepackage{graphicx}
\usepackage{natbib}
\begin{document}
\begin{frontmatter}
\title{Preparation of novel chitosan polymeric nanocomposite as an efficient material for the removal of Acid Blue 25 from aqueous environment}
\author[1]{Sivarama Krishna Lakkaboyana \corref{cors1}}
\author[1]{Khantong Soontarapa \corref{cors2}}
\author[2]{Vinaykumar}
\author[3]{Ravi Kumar Marella}
\author[4]{Karthik Kannan}
\affiliation[1]{
 organization={Department of Chemical Technology, Chulalongkorn University},
 addressline={},
 city={Bangkok},
 postcode={10330},
 country={Thailand}
}
\cortext[cors1]{Corresponding author}
\cortext[cors2]{Corresponding author}
\affiliation[2]{
 organization={Department of Biotechnology, Indian Institute of Technology Roorkee},
 addressline={},
 city={Roorkee},
 postcode={247667},
 country={India}
}
\affiliation[3]{
 organization={Department of Chemistry (H \& S), PACE Institute of Technology \& Sciences},
 addressline={},
 city={Ongole},
 postcode={523001},
 country={India}
}
\affiliation[4]{
 organization={Center for Advanced Materials, Qatar University},
 addressline={P. O 2713},
 city={Doha},
 postcode={},
 country={Qatar}
}
\begin{abstract}
A novel, sustainable chitosan polymeric nanocomposite (CS-PVA@CuO) was synthesized and subjected to the removal of acid blue 25 (AB25) from the aqueous environment. The influence of different variables such as pH, contact time, initial dye concentration, temperature, and adsorption kinetics has been examined in the batch adsorption process.
\end{abstract}
\begin{keyword}
Chitosan, PVA, CuO NWs, Adsorption, Acid Blue 25
\end{keyword}
\end{frontmatter}
% ===== MAIN DOCUMENT CONTENT =====

\section{Introduction}
A novel, sustainable chitosan polymeric nanocomposite (CS-PVA@CuO) was synthesized and subjected to the removal of acid blue 25 (AB25) from the aqueous environment. The influence of different variables such as pH, contact time, initial dye concentration, temperature, and adsorption kinetics has been examined in the batch adsorption process. The CS-PVA@CuO composite was systematically characterized by XRD, FTIR, SEM, and EDX analysis. The pseudo-first order (PFO), pseudo-second order (PSO), and intra-particle diffusion kinetics equations were used to examine the kinetic data of the adsorption process. The adsorption kinetics confirms that the PSO model was a more exact fit. Thermodynamics study typically revealed that the uptake of AB25 by the adsorbent is spontaneous and endothermic in nature. Remarkably, the results reveal the highest adsorption capacity of the CS-PVA@CuO was 171.4 mg/g at 313 K. To be specific, CS-PVA@CuO polymer nanocomposite can be effectively used as a suitable adsorbent material for the potential elimination of anionic AB25 dye from the aqueous solutions.
Dyes are considered as one of the most hazardous environmental pollutants which emerges from the discharges of the leather, paper, and textile industries \cite{1,2}. Most of the dye chemicals are toxic, harmful, non-biodegradable, and cancer-causing agents \cite{3--5}. In recent years, various techniques such as physical and chemical methods have been engaged for the potential removal of organic dyes from wastewater including adsorption \cite{6,7}, biological treatment \cite{8}, cavitation and incineration \cite{9--11}, membrane separations \cite{12,13}, advanced oxidation processes \cite{14}. Among them, adsorption process was deliberate as they are eco-friendly, more effective, economic, and for practical applications \cite{15,16}. Of late porous materials, such as graphite \cite{17}, mesoporous SiO2 \cite{18}, activated carbons (AC) \cite{17}, and have been magnificently applied for the removal of organic dyes such as methyl blue, methyl orange, Acid Blue 25 etc. However, the reported materials have low adsorption capacity and poor reusability. Accordingly, the combination of novel adsorbent materials with high dye adsorption capacity and removal capability values is still of excessive significance and a matter of practical approach.
Chitosan, a prospective bio-sorbent material, has been attracted extensive attention-holding owing to the low-cost, non-toxicity, extensive availability, biocompatibility, and biodegradability \cite{19,20}. The material has more functional groups such as -NH2 and -OH with strong coordination capability provide the more promising adsorption capability for cationic heavy metals uptake \cite{21}. Further, the amino groups present in chitosan were protonated and generate a positive charge at acidic conditions, it might capture anionic pollutants by electrostatic attraction. Nevertheless, attributing to its high solubility below pH 5, it is challenging to be recovered or separated by non-operational methods like centrifugation and filtration \cite{22}. Chitosan and polyvinyl alcohol (PVA) are co-polymerized materials largely applied in dye-containing wastewater treatment processes due to more favorable properties such as eco-friendly, high mechanical strength, more abundance, good biocompatibility, biodegradability, thermal stability, non-toxicity, adsorption properties, and cost-efficient nature. It has been stated that Chitosan incorporated with PVA composite (CS-PVA) possess good chemical and mechanical assets as a result of the intermolecular interactions between chitosan and PVA in the mixture \cite{23}. Owing to the advantages of CS, herein we have prepared a novel CS-PVA@CuO polymeric composite material, a combination of CuO nanowires with the CS-PVA.
The key objective of this work is to synthesis the novel chitosan polymeric nanocomposite beads for efficient removal of AB25 dye. Herein, CS was co-polymerized with PVA to produce an extremely stable polymeric solution. The stability of CS-PVA polymeric solution owing to the interactions, especially hydrogen bonds between the components of membrane. These cross-links can surge the extra-functional groups which exhibit a significant role to attract the dye molecules. In addition, the incorporation of CuO nanowires typically enhances the specific surface area and improves the adsorption capacity within a short period. As mentioned earlier, the polymeric material develops more hydroxyl and amine functional groups which influence the generation of CuO NWs easily on the polymeric solution without aggregation. Therefore, CS-PVA@CuO composite can provide a more specific surface area undoubtedly owing to the significant dispersion of CuO NWs. Moreover, the active Cu site of the CuO NWs typically creates surface charges which progressively improve the absorption capacity of CS-PVA@CuO composite for the potential removal of AB25 dye.

Chitosan with a degree of deacetylation of 90\% was typically purchased from SS membrane, Thailand. Polyvinyl alcohol (PVA) (with a degree of polymerization of 1750), NH4OH, HCl, and NaOH purchased from Ajax Fine Chem Pvt. Ltd., Thailand. Cu(NO3)2$\cdot$3H2O and Acid Blue 25 (AB25) procured from Sigma-Aldrich Chemicals. All the chemicals are analytical grade and utilized without other purification. The standard solutions are prepared by using double-distilled deionized water.
\subsection{2.2. Preparation of CuO nanowires}
The synthesis of CuO NWs was carried out through the chemical precipitation method. Typically, 0.1 M of Cu (NO3)2$\cdot$3H2O was dissolved in 100 mL of deionized water and kept at 50 $^\circ$C with vigorous stirring to obtain a homogeneous blue color solution. Afterward, NH4OH (aq.) is added to a Cu2+ (aq.) precursor drop by drop and simultaneously the time-temperature rises to 80 $^\circ$C, under controlled pH conditions with continuous magnetic stirring. After a few minutes, the solution is converted to a deep blue color. Subsequently, 100 mL deionized water was added to fast up the reaction and a minimal amount of NaOH is added to the solution to maintain the pH below 10. Owing to the excess of hydroxide ions in the reaction system [Cu (NH3)4]2+ complex (blue) is converted into the nanocrystalline Cu (OH)2 nanowires as a black precipitate. The obtained precipitate was filtered, washed distilled water and absolute ethanol alternatively to remove the impurities. The resultant CuO nanowires solid product was transferred into an oven and dried for 1 h at 180 $^\circ$C.
\subsection{2.3. Synthesis of CS-PVA@CuO composite}
In this work, the CS-PVA@CuO composite was prepared in weight ratios (w/w) of CS-PVA@CuO (2:2:1). Typically, 4 g of chitosan powder was dissolved in 5\% 100 mL acetic acid solution under vigorous magnetic stirring at 60 $^\circ$C until completely dissolved. In another beaker, 4 g of PVA was dissolved in 100 mL of de-ionized double distilled water under vigorous stirring at 60 $^\circ$C until completely dissolved. The above viscous solutions were mixed at 70 $^\circ$C and stirred for another 30 min. These two mixtures interacted with each other through inter-molecular hydrogen bonds and hydrophobic side chain aggregations. To the above polymeric solution, 2 g of CuO nanowires was added, and the solution was frequently stirred until the formation of a homogenous black gel. Subsequently, the homogenous black gel was transferred into a syringe (1 mm in diameter). This solution was slowly added dropwise into the dilute aqueous NaOH solution through the syringe needle, then the chitosan was triggered immediately into the form of gelatinous beads. After that, the beads were kept in the NaOH solution to solidify followed by separation and washed several times with distilled water until the pH is neutral. Later, the beads were separated from NaOH solution and washed with copious amounts of distilled water. Finally, the beads were dried in a hot air oven at 60 $^\circ$C for 12 h, stored in an airtight sealed container and kept in a desiccator until further use.
\subsection{2.4. Characterization}
The crystalline structure of CuO-NWs and CS-PVA@CuO composite was confirmed by X-ray diffraction (XRD) on a Bruker D8 Advance diffractometer. All the diffraction patterns were recorded in step-scan mode within the 2$\theta$ values ranging from 5 to 80$^\circ$ in a step interval of 0.01$^\circ$. The structural analysis of CS-PVA@CuO polymer nanocomposite was distinguished by the Fourier transform infrared spectrometer (FTIR) in a wave-number range of 4000–400 cm-1 by a resolution of 4 cm-1 in transmittance mode (FTIR, Nicolet 6700, Thermo electron Corp., USA). The morphology of CS-PVA@CuO composite sample was explored by a field emission scanning electron microscope (FE-SEM, Hitachi S4800, Tokyo, Japan) under an acceleration voltage of 15 kV. The elementary composition of the CuO-NWs and CS-PVA@CuO composite was obtained by the energy dispersive spectrometer (EDS, SU3500).
\subsection{2.5. Adsorption experiments}
In this study, AB 25 is typically chosen as a model pollutant, and the concentrations of AB 25 are obtained by the linear regression equations (derived by plotting its calibration curve). The AB 25 dye adsorption capacity of the CS-PVA@CuO composite was confirmed at the time interval in the range of 1–60 min for 25–100 mg/L at room temperature and it was found that equilibrium was established after 30 min. The effect of reaction variables such as dye initial concentration, contact time, and adsorption temperature on the effective removal of AB 25 was studied in the batch experiments. To properly estimate and control the kinetic, thermodynamic, and isotherm parameters of the sorption process, all adsorption experiments are carried out using a shaker with a horizontal water bath at a rotating speed of 200 rpm. The potential effect of contact time was executed at different AB 25 dye concentrations at 25–100 mg/L and 0–60 min, respectively. The adsorption equilibrium isotherms are accomplished with various initial concentrations of AB 25 (100–300 mg/g) at different temperatures (30, 35, and 40 $^\circ$C). These flasks were shaken constantly for 60 min with 200 rpm. After the reaction had reached the adsorption equilibrium, the composite and unabsorbed dye solution are separated by centrifuge followed by filtration. Later, the remaining dye concentration was accurately estimated by using a UV–Vis spectrophotometer at 610 nm. All the adsorption experiments are performed in triplicate and the average values are typically provided.
The amount of AB 25 on the CS-PVA@CuO composite (qe) and removal efficiency (R) is predicted at the equilibrium conditions from the following Eqs. (1) and (2), respectively.
The specified amount of AB 25 adsorbed by CS-PVA@CuO at a time ‘t’ (qt) was calculated using Eq. (1),
qt = \frac{(Co - Ct) \cdot V}{W}

In Eq. (1), $C_o$ and $C_t$ (mg/L) represents the initial concentration and concentration at time $t$, respectively; $V$ (L) is the volume of AB25 solution, and $W$ (mg) is the mass of CS-PVA@CuO.
The removal percentage of AB25 onto CS-PVA@CuO was evaluated using Eq. (2),
\%Removal $= \frac{C_o-C_e}{C_o} \cdot 100$
where $C_e$ (mg/L) denotes the equilibrium concentration of AB25 in the solution.
\section{Result and discussion}
\subsection{Characterization of CS-PVA@CuO composite}
The powder X-ray diffraction patterns of the CS-PVA@CuO nano-composite and CuO NWs are shown in Fig. \ref{fig:1}. The samples have shown the diffractions at $2\theta$ values of 35.8, 38.5, 48.7, 58.2 and 66.5$^\circ$, promptly confirming the active presence of CuO (JCPDS No. 05-661) \cite{ref24}. The average crystallite size of synthesized CuO nanowires (~17.5 nm) was calculated by Debye-Scherrer's. As shown in Fig. \ref{fig:1b}, the broad reflection around $2\theta$ value of 19.6$^\circ$ with the lattice plane (220) suggests the active presence of chitosan (JCPDS No. 089-2529) in the CS-PVA polymer matrix \cite{ref25}. The broad x-ray diffractions with high intensity at 35.8 and 38.5$^\circ$ confirm the formation of CS-PVA@CuO nanocomposite. In addition, the average crystalline size of CuO predictable to be around 10.7 nm which can be attributed to the high dispersion and less aggregation of CuO NWs in the CS-PVA@CuO nanocomposite \cite{ref26}.

The powder X-ray diffraction patterns of the CS-PVA@CuO nano-composite and CuO NWs are shown in the Fig. 1. The samples have shown the diffractions at $2\theta$ values of 35.8, 38.5, 48.7, 58.2 and 66.5$^\circ$, promptly confirming the active presence of CuO [24]. The average crystallite size of synthesized CuO nanowires (~17.5 nm) was calculated by Debye-Scherrer's. As shown in Fig. 1b, the broad reflection around $2\theta$ value of 19.6$^\circ$ with the lattice plane (220) suggests the active presence of chitosan in the CS-PVA polymer matrix [25]. The broad x-ray diffractions with high intensity at 35.8 and 38.5$^\circ$ confirm the formation of CS-PVA@CuO nanocomposite. In addition, the average crystalline size of CuO predictable to be around 10.7 nm which can be attributed to the high dispersion and less aggregation of CuO NWs in the CS-PVA@CuO nanocomposite [26].
FTIR spectra of CS-PVA, CuO NWs, CS-PVA@CuO, and AB25 are shown in Fig. 2. As shown in Fig. 2a, the absorption band from 3450 to 3200 cm$^{-1}$ is assigned to O--H and N--H stretching vibrations. Subsequently, the band at 2923 cm$^{-1}$ is associated with C--H stretching vibration. It can be observed that O--H band slightly broadened in CS-PVA due to the intermolecular hydrogen bonding between the oxygen of PVA and hydroxyl groups of chitosan. The band at 1650 cm$^{-1}$ is attributed to the C--O stretching of the acetyl group (amide I). The band at 1548 cm$^{-1}$ is assigned to the N--H bending and stretching (amide II) [27]. In notable addition, the absorption band at 3450--3200 cm$^{-1}$ confirms that the O--H stretching band overlap with the N--H bands of amine and amide in the CS-PVA blended polymer [28]. The sharp C--O stretching vibration band at 1062 and a small peak 1159 cm$^{-1}$ promptly confirm the interaction between the CS and PVA crosslinking by $\beta(1 \rightarrow 4)$ glycosidic bonds [29--31]. The peaks between 1750 and 1735 cm$^{-1}$ appeared in CS-PVA are due to the C--O and C--O stretching vibrations from acetate groups in the PVA [32].
The FTIR results of CS-PVA@CuO (Fig. 2c) show the broad peak between 3900 and 3570 cm$^{-1}$ due to O--H and N--H groups stretching vibration, another peak at 2920 cm$^{-1}$ for C--H aliphatic stretching, and a band at 1750 cm$^{-1}$ due to acetyl amide group present in chitosan structure [33]. A strong absorption peak at 1420 cm$^{-1}$ assigned to the symmetrical stretching vibration of COO$^{-}$ group. The spectrum in the range of 1158 cm$^{-1}$ can be attributed to the C--O stretching vibrations. The absorption band of the NH$_2$ and OH groups at 3420 cm$^{-1}$ are shifted to 3445 cm$^{-1}$. The characteristic peaks observed in the range of 525--587 cm$^{-1}$ can be attributed to the stretching vibration of Cu--O bond in monoclinic CuO [34]. On comparison of the FTIR spectra of CS-PVA with CS-PVA@CuO, the bands corresponding to the hydroxyl, amine and amide groups are shifted in the spectrum of hybrid material. The shift in the IR bands is established because of the interaction between the CS-PVA and CuO NWs.
The FTIR spectra after the adsorption of AB25 on CS-PVA@CuO nanocomposite are illustrated in Fig. 2d. As shown in the figure, the FTIR peaks at 3243, 2941, 1663, 1575, 1454, 1412, 1315, 1023, and 900 cm$^{-1}$ are established in the spectra (Fig. 2d). Assessment of FTIR spectra of AB25 dye loaded CS-PVA@CuO nanocomposite with unloaded CS-PVA@CuO nanocomposite revealed that the significant changes in --OH and --NH$_2$ vibration bands which decreased drastically from 3900 to 3243 cm$^{-1}$. Similarly, the peak at 1750 cm$^{-1}$ assigned to the acetyl amide group present in chitosan also shifted to a lower wavelength 
\begin{figure}[htbp]
\centering
\includegraphics[width=0.8\linewidth]{images/figure1.png}
\caption{Powder XRD of CuO NWs and CS-PVA@CuO nanocomposite}
\label{fig:1}
\end{figure}
\begin{figure}[htbp]
\centering
\includegraphics[width=0.8\linewidth]{images/figure2.png}
\caption{FTIR Spectra of (a) CS-PVA, (b) CuO NWs, (c) CS-PVA@CuO nanocomposite, and (d) after AB25 adsorption on CS-PVA@CuO}
\label{fig:2}
\end{figure}

\section{2–O3-S-Dye}
The surface morphology of CS-PVA@CuO nanocomposite was analyzed by SEM analysis and the results are illustrated in Fig. 3. From the figure, the composite materials were obtained to be rough surfaces with copper nanorods. The SEM-EDS analysis is one of the advantageous tools to estimate the fundamental characteristics of the adsorbent. The elementary distribution of CS-PVA@CuO was calculated by SEM-EDS analysis and demonstrated in Fig. 4. The composite weight percentage and atomic percentage of all the atoms are shown in Table 1. The major atoms are C, O, and Cu emission peaks indicate the existence of these functional groups in the polymer matrix. The elemental composition of pure CuO nanorods (Fig. 4b) was clearly obtained from EDS spectra, and results indicated the presence of Cu and O. In general, the EDS analysis predicts the relative percentage of desired elements present in the specimen. Therefore, after the addition of CuO in the CS-PVA polymer matrix, the relative \% of O atom increases with the expense of C percentage and Cu atom have appeared in the composite material. These consistent results unanimously indicate the successful formation of CS-PVA@CuO.
\subsection{3.2. Removal of Acid Blue 25}
\subsubsection{3.2.1. Effect of initial pH on the AB25 adsorption}
The adsorption process and its capacity typically depend on the pH of dye solution owing to the several factors such as ionization of functional groups in acidic/basic solutions. Essentially, the elimination of AB25 was under the influence of initial solution pH, and the adsorbent surface charge (denoted by pHpzc), therefore dye removal capacity may be increased or decreased.
The impact of solution pH on the elimination of AB25 was measured in the pH range from 2.0 to 10.0 and the results are graphically represented in Fig. 5. As shown in figure, the adsorption percentage decreased from 59.1 to 47.38\% with an increase in the pH value from 8.0 to 10.0. At a lower pH value, the number of protons available is limited since the functional groups in the material are protonated at lower pH. Consequently, the optimum pH was observed to be at acidic pH solution which can benefit the adsorption of the negatively charged anionic AB25 dye \cite{35}. Nevertheless, it should be noted that the adsorption capacity of CS-PVA@CuO nanocomposite at a pH of 2.0 and 4.0 is lower in comparison with the pH value of 6.0 (98.86\%). The CS-PVA@CuO nanocomposite was unstable at lower pH values and even gets demolished at the extreme acidic conditions. The anionic AB25 dye molecules are
Fig. 3. FE-SEM image of CS-PVA@CuO nanocomposite.
(1663 cm$^{-1}$). On contrary, the CuO peaks (1545 and 1384 cm$^{-1}$) shifted to the higher wavelength (1575 and 1412 cm$^{-1}$) after adsorption of AB25 onto CS-PVA@CuO nanocomposite. This observation reveals that the hydroxyl/amine and amide functional groups are strongly involved in the AB25 dye adsorption onto the CS-PVA@CuO nanocomposite surfaces. Thus, considering the above FTIR results, we have proposed the following adsorption mechanism.
i. The AB25 dye dissociated in aqueous solution
Dye$-$SO$_3$Na $\rightarrow$ Dye$-$SO$_3$$^{-}$ $+$ Na$^{+}$
ii. At acidic pH, the $-$NH$_2$ and $-$OH groups present in the CS-PVA@CuO
get protonated
R$-$NH$_2$ $+$ H$_3$O$^{+}$ $\rightarrow$ R$-$NH$_3^{+}$ $+$ H$_2$O
R$-$OH $+$ H$_3$O$^{+}$ $\rightarrow$ R$-$OH$_2^{+}$ $+$ H$_2$O

R--OH $+$ H$_3$O$^+$ $\rightarrow$ R--OH$^+$ $\rightarrow$ 
2 $+$ H$_2$O
iii. In the final step, adsorption process takes place by the electrostatic attraction among the positively charged nitrogen atom/hydroxyl molecules and the Dye-SO$_3$$^-$ anion.
\begin{figure}[htbp]
\centering
\includegraphics[width=0.8\linewidth]{images/figure4.png}
\caption{SEM-EDS spectrum of CS-PVA@CuO nanocomposite}
\label{fig:4}
\end{figure}
\cite{S.K.Lakkaboyana, K.Soontarapa, Vinaykumar et al.}
\begin{table}[htbp]
\centering
\caption{SEM-EDS analysis of CS-PVA@CuO}
\label{tab:1}
\begin{tabular}{llll}
\hline
S. no. & Element & Weight\% & Atomic\% \\
\hline
1 & C & 33.36 & 27.60 \\
2 & O & 0.81 & 13.06 \\
3 & Al & & \\
4 & Si & & \\
5 & Cu & & \\
6 & Total & 35.18 & \\
\hline
\end{tabular}
\end{table}

\section{Adsorption Kinetics}
The adsorption capacity of AB25 with the extension of contact time until the adsorption equilibrium time 30 min. The uptake rates of AB25 gradually improved from 23 to 40\% within the first 15 min, and then sharply increased at the end. In the same way, the removal efficiency also rose from 87.92 to 96.52\%, which could be endorsed to the plenty of active sites present on the adsorbents. Additionally, the adsorption process developed swiftly at the initial stages as shown in the slope of the curve in Fig. 6, and then became stable after reaching the maximum value. This is most likely due to the adsorption acted on the microsphere surface initially, i.e., the mass transfer from the microsphere surface of AB25 to the solution happens rapidly. Then, AB25 infiltrated into the interior of the microsphere from the surface, which is slower than the previous stage \cite{39}. The number of channels and pores in the microsphere decides two infiltration rates and absorption rates in the later stage \cite{40}.
Typically, in adsorption process, the kinetic parameters perform a dynamic role because it provides information about the mass transfer of molecules/ions from the liquid phase to the adsorbent's surface. In addition, it also gives the information about the adsorption mechanism of AB25 molecules onto CS-PVA@CuO nanocomposite. The adsorption kinetics at various initial AB25 concentrations are typically shown in Fig. 6(a--c). The experimental kinetic data including PFO, PSO, Elovich and intraparticle diffusion models were studied \cite{41,42}. By means of the earlier models, we can evaluate the kinetic data for AB25 adsorption to find out a reliable model for expressing the experimental $q_e$ value. The kinetic equations are as shown below,

\begin{figure}[htbp]
\centering
\includegraphics[width=0.8\linewidth]{images/figure5.png}
\caption{Effect of solution pH}
\label{fig:5}
\end{figure}
The PFO model : $q_{t} = q_{e} 1- \exp -K_{1} t$
Where the nonlinear form of PSO : $q_{t} = \frac{K_{2}q_{2e} t}{1 + q_{e}K_{o}t}$
attracted to the leading positively charged surface and this remark recommends that the adsorption of AB25 onto these CS-PVA@CuO composite depends mainly on ionic interaction \cite{36,37}. Moreover, the CS-PVA@CuO nanocomposite has pHpzc of 6.2, hence the adsorption removal percentage is higher at a pH $<$ 6.2. The experimental results confirmed the optimum pH was in the range of 4.0--6.0, for the maximum removal of AB25 dye. With a further rise in the pH value above 6.0, there is a gradual decrease in the percentage elimination of AB25 dye due to the competition between hydroxyl ions and anionic sulfonic groups. This phenomenon typically decreases the total number of adsorption sites present in the AB25 as reported elsewhere \cite{38}.

attracted to the leading positively charged surface and this remark recommends that the adsorption of AB25 onto these CS-PVA@CuO composite depends mainly on ionic interaction \cite{36,37}. Moreover, the CS-PVA@CuO nanocomposite has pHpzc of 6.2, hence the adsorption removal percentage is higher at a pH $<$ 6.2. The experimental results confirmed the optimum pH was in the range of 4.0--6.0, for the maximum removal of AB25 dye. With a further rise in the pH value above 6.0, there is a gradual decrease in the percentage elimination of AB25 dye due to the competition between hydroxyl ions and anionic sulfonic groups. This phenomenon typically decreases the total number of adsorption sites present in the AB25 as reported elsewhere \cite{38}.
\subsection{Adsorption kinetics}
The experimental data to estimate the adsorption kinetics was executed under initial dye concentrations at 50, 75, and 100 mg/L, respectively. As shown in Fig. 6, it can be identified that hike in the 
``K2'' represents the PSO constant (g/mg$\cdot$h), ``t'' is the time (h), ``qe'' and ``qt'' indicates the quantity of AB25 (g/mg) adsorbed on the surface of CS-PVA@CuO composite at equilibrium and at time ``t'' respectively.
The Elovich kinetic model can be written as:
\begin{equation}
qt = \frac{1}{\beta} \ln \alpha\beta + \frac{1}{\beta} \ln t
\label{eq:elovich}
\end{equation}
where $\alpha$ (g/mg) and $\beta$ (g/mg) are the parameters of the Elovich rate equation.
The PFO, PSO, and Elovich kinetic model data for AB25 was shown in Fig. 6(a--c), and the kinetic parameters are shown in Table 2. The obtained results show the R2 values (0.9981--0.9932) of the PSO kinetic model are higher than pseudo-first-order model in the range of 0.9988--0.9998. In addition, the calculated qe values in PSO kinetic model are more promising with the experimental values (qe). So, considering the prior information it can be concluded that the adsorption of AB25 is fitted to the chemisorption \cite{43}. In addition, the intra-particle diffusion calculation plots don't pass through the origin, which suggests the rapid reaction had occurred in the adsorption processes. The limited distance of R2 values (Table 3 and Fig. 7) from 
\begin{figure}[htbp]
\centering
\includegraphics[width=0.8\linewidth]{images/figure1.png}
\caption{Figure caption}
\label{fig:label}
\end{figure}

\begin{figure}[htbp]
\centering
\includegraphics[width=0.8\linewidth]{images/figure2.png}
\caption{Exp.qePsedo first orderPsedo second orderElovich}
\label{fig:label}
\end{figure}

\begin{figure}[htbp]
\centering
\includegraphics[width=0.8\linewidth]{images/figure3.png}
\caption{(a)}
\label{fig:}
\end{figure}
\begin{figure}[htbp]
\centering
\includegraphics[width=0.8\linewidth]{images/figure4.png}
\caption{(b)}
\label{fig:}
\end{figure}
\begin{figure}[htbp]
\centering
\includegraphics[width=0.8\linewidth]{images/figure5.png}
\caption{Exp.qePsedo first orderPsedo second orderElovich}
\label{fig:}
\end{figure}
\begin{figure}[htbp]
\centering
\includegraphics[width=0.8\linewidth]{}
\caption{(c)}
\label{fig:}
\end{figure}

\begin{figure}[htbp]
\centering
\includegraphics[width=0.8\linewidth]{figure.png}
\caption{Exp.qePsedo first orderPsedo second orderElovich}
\label{fig:label}
\end{figure}

\begin{figure}[htbp]
\centering
\includegraphics[width=0.8\linewidth]{figure.png}
\caption{Pseudo-first order, Pseudo-second order, Elovich model Kinetics plots at different concentration of AB 25 (a) 50, (b) 75, and (c) 100 mg/L using CS-PVA@CuO nanocomposite.}
\label{fig:6}
\end{figure}
\begin{table}[htbp]
\centering
\caption{Estimated kinetic model parameters for Acid Blue 25 dye (AB25) adsorption using CS-PVA@CuO at different concentrations.}
\label{tab:2}
\begin{tabular}{llll}
\hline
Concentration & 50 mg/L & 75 mg/L & 100 mg/L \\
\hline
2t / (1 + qek2t) & & & \\
qt = k2qeaqek2R2SENSDARE & 32.75 & 0.003 & 0.988 \\
q = qe(1 - exp(-k1t))qek1R2SENSDARE & 1.698 & 6.242 & -4.994 \\
\hline
\end{tabular}
\end{table}
S.K. Lakkaboyana, K. Soontarapa, Vinaykumar et al. International Journal of Biological Macromolecules 168 (2021) 760–768

26.57
0.083\\
0.983\\
1.473\\
4.695\\
$-3.756$
bqt = $(1/\\beta)$ln$(\\alpha\\beta)+1/\\beta)$lnt$\\beta$R$^2$SENSDARE
0.131\\
5.195\\
0.938\\
1.310\\
4.307\\
$-3.446$
41.64
0.004\\
0.994\\
1.187\\
0.934\\
$-0.747$
36.37
0.135\\
0.996\\
0.974\\
0.031\\
$-0.025$
0.132
23.50
0.913\\
2.589\\
2.193\\
$-1.754$
51.79
0.003\\
0.998\\
0.811\\
1.830\\
1.464
44.82
0.134\\
0.988\\
2.145\\
0.973\\
0.778
0.105
26.77
0.978\\
1.574\\
0.279\\
$-0.224$

\subsection{Equilibrium adsorption isotherms}
From the Fig. 8, the adsorption capacities of CS-PVA@CuO increased with increase in the initial dye concentrations. The higher initial concentration aids the dye molecules easier to overcome the mass transfer resistance between the AB25 and the as-prepared CS-PVA@CuO composite \cite{42}. To determine the temperature effect on AB25 dye adsorption, the adsorption experiment was performed at different temperatures (303, 308, and 313 K). In this study, the adsorption data for AB25 on the CS-PVA@CuO was implemented to investigate adsorption behavior by the Langmuir and Freundlich isotherm models. Additionally, a dimensionless constant called equilibrium parameter (RL) is commonly used to forecast whether an adsorption system is favorable or unfavorable, as shown in Eqs. (6), (7), and (8) \cite{42}. These two adsorption isotherms models are shown as below:
Langmuir, $q = \frac{K_{L}q_{m}C_{e}}{1 + K_{L}C_{e}}$
Freundlich, $q = K_{f} C_{e}^{n_{e}}$
where ``q'' is the quantity of adsorbed AB 25 at equilibrium settings (mg/g), $C_{e}$ represents the AB 25 equilibrium concentration in solution (mg/L). The adsorption isotherms constant was estimated employing
\begin{table}[htbp]
\centering
\caption{Intra-particle diffusion model for Acid blue 25 dye (AB25) adsorption using CS-PVA@CuO at different concentrations}
\label{tab:label}
\begin{tabular}{llll}
\hline
Linear portion & First & Second & Constant \\
\hline
$C_{0} = 50$ mg/L & & & \\
$C_{0} = 75$ mg/L & & & \\
\hline
\end{tabular}
\end{table}

C0 = 100 mg/L
KP1 (mg/g min$^{0.5}$) & C1 (mg/g) & R2 & KP2 (mg/g min$^{0.5}$) & C2 (mg/g) & R2 \\
\hline
5.177 & 1.221 & 0.946 & 1.800 & & \\
13.01 & & 0.999 & & & \\
8.259 & 0.011 & 0.994 & 1.645 & & \\
24.99 & & 0.984 & & & \\
10.13 & & 0.432 & 0.998 & 0.205 & \\
31.25 & & 0.966 & & & \\
the experimental data attained from nonlinear regression through excel-solver software. Adsorption isotherm non-linear fitting results are illustrated in Fig. 8 and fitting parameters are highlighted in Table 4. From Fig. 8 the other parameters are different isotherm constants were calculated, this parameter can be influenced by regression of the experimental data. Traditionally, the two-parameter equation models are Langmuir and Freundlich, which were extensively used than the three-parameter equation models. From Table 4, the correlation coefficient is more fitting the linear form to the Langmuir model is indeed higher than the Freundlich isotherm model under all temperature conditions. These outcomes indicated that the adsorption processes were monolayer coverage of AB25 on the accessible surface of CS-PVA@CuO composite with the maximum capacity is 171.4 mg/g at 313 K. Additionally, the Langmuir constant, KL increased with temperature, it indicates the affinity for the binding of AB25 and the results confirm that high value of KL represents higher affinity at a more elevated temperature. From the Langmuir isotherm, it was established that the monolayer maximum sorption capacity (qmax) is increased from 165.1 mg/g to 171.4 mg/g with the temperature of the system increasing from 303 to 313 K, which indicate the adsorption process is endothermic in nature. The expectable model parameters with the correlation coefficient (R2) and standard error (S.E), nonlinear chi-square test ($\chi^2$), and root mean square error (RMSE) for the various models are summarized in Table 4. From the two models, Langmuir model delivered better fitting for the isotherm data in terms of R2, SE, RMSE, $\chi^2$, and RL1 values. The dimensionless equilibrium constant RL also determined for the Langmuir model. It proposes the possibility of the adsorption process being irreversible (RL = 0), favorable (0 < RL < 1), linear (RL = 1), or unfavorable (RL > 1). This separation factor (RL) is expressed by the following equation,
RL = $\frac{1}{1 + \text{KL} \text{C0}}$ \label{eq:RL}
where KL is the Langmuir constant, while C0 is the initial concentration.

where ``KL'' is the Langmuir constant, while ``Co'' is the initial concentration.
From the Table 4, all the RL values within the range of 0 $<$ RL $<$ 1 confirmed that the AB25 dye ions are more favorably adsorbed on CS-PVA@CuO composite.
From the Freundlich isotherm model to define the KF ((mg/g)/(mg/L)$^{1/n}$) and n are constants these indicate the capacity and intensity of the adsorption, respectively. The reciprocal of constant ``n'' represents the reference heterogeneity factor (1/n). These constants are obtained
MB-50 mg/L
MB-75 mg/L
MB-100 mg/L
Kp1
Kp2
50
40
30
20
10
0
"g/gm"
(
e
q

Fig. 7. Intraparticle diffusion model for the adsorption of AB25 over CS-PVA@CuO.
765
S.K. Lakkaboyana, K. Soontarapa, Vinaykumar et al.
International Journal of Biological Macromolecules 168 760–768
Fig. 8. Freundlich and Langmuir model isotherm plots for the adsorption of AB25 dye on CS-PVA@CuO composite at different temperatures (a) 30, (b) 35, and (c) 40 $^\circ$C.
from the intercept and slope of the line accomplish through plotting logqe versus log Ce and the values are tabulated in Table 4. Moreover, the low values of the correlation coefficient demonstrate the poor agree-ment of the experimental data. The small “n” value is the sign of good adsorption at elevated temperature and more suitable for the less fitting of the experimental data at a lower temperature. In addition, the higher value of KF designates the higher capacity of adsorbents at higher tem-peratures. These results confirm the adsorption process of AB25 onto CS-PVA@CuO composite is more encouraging at a higher temperature. For the authentication and quality of fitting the obtained data was ap-plied to the various error functions with the correlation coefficient (R2). From Table 4, the results confirm the lowest value of S.E, $\chi^{2}$, and RMSE with higher values of R2 for Langmuir model in the demonstrated experimental values. This confirms the Langmuir isotherm model finds the best fit for the experimental values.
The thermodynamic parameters namely the standard Gibbs energy change ($\Delta$G$^\circ$), enthalpy change ($\Delta$H$^\circ$), and entropy change ($\Delta$S$^\circ$) of the adsorption of AB25 dye onto the CS-PVA@CuO were assessed to following equations.

\begin{equation}
\Delta G = -RT \ln K_d
\label{eq:1}
\end{equation}
The Gibbs free energy change, $\Delta G^{\circ}$, is the fundamental criteria of the spontaneity of a process. The standard Gibbs free energy was expressed at different temperatures according to the following equation;
\begin{equation}
\Delta G = \Delta H - T\Delta S
\label{eq:2}
\end{equation}
The change in enthalpy ($\Delta H^{\circ}$) and entropy ($\Delta S^{\circ}$) are calculated from the following equations,
\begin{table}[htbp]
\centering
\caption{Estimated isotherm parameters for Acid blue 25 dye (AB25) adsorption using CS-PVA@CuO as adsorbent at different constant temperatures}
\label{tab:1}
\begin{tabular}{llll}
\hline
Adsorption & $1/n$ & $q = K_f C_e^{1/n}$ & $R^2$ \\
\hline
$30 ^{\circ}$C & 0.399 & $32.6$ & $0.987$ \\
$35 ^{\circ}$C & 0.628 & $4.997$ & $0.980$ \\
$40 ^{\circ}$C & 0.082 & $4.080$ & $0.995$ \\
\hline
Adsorption & $K_L$ (L/mg) & $q_m$ (mg/L) & $R^2$ \\
\hline
$30 ^{\circ}$C & 0.112 & $165.1$ & $0.995$ \\
$35 ^{\circ}$C & 0.029 & $5.204$ & $0.983$ \\
$40 ^{\circ}$C & 0.082 & $4.250$ & $0.981$ \\
\hline
$R_L$ & $37.4$ & $0.398$ & $0.983$ \\
\hline
\end{tabular}
\end{table}

where R is the universal gas constant (8.314 $\times$ 10$^{-3}$ kJ/mol$\cdot$K), ``T'' represents the absolute temperature in K, and MW is the adsorbate molecular weight. The value of Kd can be derived by multiplying the Langmuir constant ``b''. The values of thermodynamic parameters are typically arranged in Table 5. $\Delta$G$^{\circ}$ indicates the feasibility and spontaneous nature of the sorption processes under standard conditions. From Table 5, the $\Delta$G$^{\circ}$ exhibit more negative values with increase in the temperature, which reveals that adsorption of AB25 is a spontaneous process and thermodynamically more favorable at higher temperature \cite{44}. Furthermore, the $\Delta$G$^{\circ}$ values are less than 60 kJ/mol confirm the adsorption of AB25 on the composite was chemisorption process. The value of $\Delta$S$^{\circ}$ was found to be positive establish the increased the randomness in the solid-liquid interface interaction during the adsorption process. The positive values of $\Delta$H$^{\circ}$ point out that the more interaction of the CS-PVA@CuO composite with AB25. The attained results are confirming the adsorption process is endothermic nature. In general, the magnitude of $\Delta$H$^{\circ}$ for physisorption is 2.1--20.9 kJ/mol and for chemisorption is 20.9--418.4 kJ/mol. Herein, the $\Delta$H$^{\circ}$ value of this adsorption process equals to 11.212 kJ/mol for CS-PVA@CuO composite, indicating that the adsorption is physisorption process. The positive value of $\Delta$S$^{\circ}$ reflects the affinity of the adsorbent material for AB 25 as well as increase of randomness at solid--solution interface during adsorption \cite{44,45}.
\subsection{Assessment of adsorption capacities of various adsorbents for the removal of AB 25}
A comparison of various reported adsorbents with the present adsorbent system for the AB25 dye is summarized in Table 6. Han et al. carried out a reaction at 325 K and pH = 3.0 using natural sepiolite with a maximum AB25 adsorption capacity (qe) of 87.5 mg/g (Table 6, entry 1). Krishna et al. investigated the use of Indian jujuba seed powder (IJSP) to obtain the qe of 54.95 mg/ g (Table 6, entry 2). Brahmi et al. applied the activation carbon prepared from wild date stones (WDS-AC) to get qe value of 181.59 mg/g (Table 6, entry 3). Auta et al. reported the use of waste tea activated carbon (WTAC) prepared by chemical activation of potassium acetate exhibited a maximum monolayer adsorption 
\begin{table}[htbp]
\centering
\caption{Thermodynamic parameters}
\label{tab:5}
\begin{tabular}{lll}
\hline
T (K) & KL (L/mg) \\
\hline
\end{tabular}
\end{table}

KL (L/mg)
$\Delta$G$^{\circ}$ (kJ/mol)
$\Delta$H (kJ/mol)
$\Delta$S (kJ/mol$\cdot$K)
303 308 313
0.11 0.15 0.17
$-$27.09 $-$28.27 $-$29.08
16.66
0.146
S.K. Lakkaboyana, K. Soontarapa, Vinaykumar et al.
International Journal of Biological Macromolecules 168 (2021) 760--768
\begin{table}[htbp]
\centering
\caption{Comparison of maximum adsorption capacities of various adsorbents for the removal of AB 25 dye}
\label{tab:label}
\begin{tabular}{lll}
\hline
S. no. & Adsorbent & qe (mg/g) \\
\hline
Reference
\end{tabular}
\end{table}
Acquisition of data: Sivarama Krishna Lakkaboyana, Analysis and/or interpretation of data: Sivarama Krishna Lakkaboyana, Ravi Kumar Marella, Karthik Kannan

\section{Conclusions}
In summary, a simple, eco-friendly, effective CS-PVA@CuO composite established to gently remove AB 25 from the aqueous solution. The prepared CS-PVA@CuO composite displayed the significant dispersion of CuO NWs in the organic matrix. The CS-PVA@CuO has demonstrated to be an effective adsorbent for fast removal of AB25 with a high adsorption capacity of 171.4 mg/g. In addition, CS-PVA@CuO composite removal capacity displays a considerable increase with an increase in temperature. Thermodynamic constants ($\Delta$G$^\circ$, $\Delta$H$^\circ$, and $\Delta$S$^\circ$) show that the chemical adsorption process is endothermic and spontaneous in nature. The experimental results prove that chitosan polymeric nanocomposite can be considered as a promising material for the removal of AB25 from the aqueous environment. Furthermore, it is unharmful to the ecosystem and aquatic environments particularly compared to metal oxide nanoparticles and nanocomposite.
CRediT authorship contribution statement
Conception and design of study: Sivarama Krishna Lakkaboyana,
Khantong Soontarapa
Khantong Soontarapa, Vinaykumar
Revising the manuscript critically for important intellectual content:
Sivarama Krishna Lakkaboyana, Ravi Kumar Marella, Karthik Kannan
Approval of the version of the manuscript to be published:
Sivarama Krishna Lakkaboyana, Khantong Soontarapa, Vinaykumar, Ravi Kumar Marella, Karthik Kannan 
capacity (qe) of 203.34 mg/g (Table 6, entry 4). Hanafiah et al. claimed the potential use of base treated Shorea dasyphylla (BTSD) saw dust for AB25 removal with a qe value of 24.39 mg/g at 300 K (Table 6, entry 5). Ferrero et al. studied batch adsorption of AB 25 onto ground hazelnut shells in comparison with sawdust of various wood species (Table 6, entries 6–10). Kooh et al. reported the use of plant-based materials, water lettuce (WL), tarap peel (TP), and cempedak peel (CP) as possible adsorbents for AB25 (Table 6, entries 11, 12, and 13). Mckay et al. studied the sorption of AB25 onto wood with a mild equilibrium sorption capacity (qe) of 6.4 mg/g (Table 6, entry 14). Krishna et al. investigate the biosorption of AB25 onto Bengal gram fruit shell (BGFS) and explore the effect of cationic surfactant, cetyltrimethyl ammonium bromide (CTAB). It can be observed that the presence of CTAB in the reaction medium improved the qe of AB25 to 166.6 mg/ g, which is 5.7 times greater than qe in the absence of CTAB (Table 6, entries 15 and 16). Likewise, Indonesian natural zeolite and CTAB modified zeolite followed a similar trend with qe value of 64.2 and 112.44 mg/g was obtained for the zeolite and zeolite-CTAB respectively.
Remarkably, the performance results proved that the CS-PVA@CuO composite has an excellent adsorption capacity compared with the reported literature (Table 6, entry 20).

Ravi Kumar Marella, Karthik Kannan
Acknowledgments
The author Lakkaboyana Sivarama Krishna is grateful to the graduate school and the Thailand Research Fund (IRG578001), Chulalongkorn University for providing financial support as postdoctoral fellowship under Rachadapisaek Sompote Fund.

% ===== REFERENCES =====

\begin{thebibliography}{99}

ibitem{ref1} M.C.S. Reddy, L. Sivaramakrishna, A.V. Reddy, The use of an agricultural waste material, Jujuba seeds for the removal of anionic dye (Congo red) from aqueous medium, J. Hazard. Mater. 203-204 (2012) 118–127.

ibitem{ref2} V.K. Gupta, R. Kumar, A. Nayak, T.A. Saleh, M.A. Barakat, Adsorptive removal of dyes from aqueous solution onto carbon nanotubes: a review, Adv. Colloid Interf. Sci. 193-194 (2013) 24–34.

ibitem{ref3} P.-T. Wang, Y.-H. Song, H.-C. Fan, L. Yu, Bioreduction of azo dyes was enhanced by in-situ biogenic palladium nanoparticles, Bioresour. Technol. 266 (2018) 176–180.

ibitem{ref4} L.S. Krishna, A.S. Reddy, A. Muralikrishna, W.Y.W. Zuhairi, H. Osman, A.V. Reddy, Utilization of the agricultural waste (Cicer arientinum Linn fruit shell biomass) as biosorbent for decolorization of Congo red, Desalin. Water Treat. 56 (2015) 2181–2192.

ibitem{ref5} L. Yu, M.-Y. Cao, P.-T. Wang, S. Wang, Y.-R. Yue, W.-D. Yuan, W.-C. Qiao, F. Wang, X. Song, Simultaneous decolorization and biohydrogen production from Xylose by GS-4-08 in the presence of azo dyes with sulfonate and carboxyl groups, Appl. Environ. Microbiol. 83 (2017) e00508–e00517.

ibitem{ref6} M. de Graaff, M.F.M. Bijmans, B. Abbas, G.-J.W. Euverink, G. Muyzer, A.J.H. Janssen, Biological treatment of refinery spent caustics under halo-alkaline conditions, Bioresour. Technol. 102 (2011) 7257–7264.

ibitem{ref7} W. Qiao, J. Chu, S. Ding, X. Song, L. Yu, Characterization of a thermo-alkali-stable laccase from Bacillus subtilis cjp3 and its application in dyes decolorization, J. Envi-ron. Sci. Health A 52 (2017) 710–717.

ibitem{ref8} P.S. Bauerlein, T.L. ter Laak, R.C.H.M. Hofman-Caris, P. de Voogt, S.T.J. Droge, Removal of charged micropollutants from water by ion-exchange polymers – effects of competing electrolytes, Water Res. 46 (2012) 5009–5018.

ibitem{ref9} P.S. Bauerlein, T.L. ter Laak, R.C.H.M. Hofman-Caris, P. de Voogt, S.T.J. Droge, Removalof charged micropollutants from water by ion-exchange polymers – effects of com-peting electrolytes, Water Res. 46 (2012) 5009–5018.

ibitem{ref10} S.S. Khan, Enhancement of visible light photocatalytic activity of CdO modified ZnO nanohybrid particles, J. Photochem. Photobiol. B Biol. 142 (2015) 1–7.

ibitem{ref11} H.R. Mortaheb, A. Zolfaghari, B. Mokhtarani, M.H. Amini, V. Mandanipour, Study onremoval of cadmium by hybrid liquid membrane process, J. Hazard. Mater. 177(2010) 660–667.

ibitem{ref12} Y. Zhang, Y. Xiong, Y. Tang, Y. Wang, Degradation of organic pollutants by an inte-grated photo-Fenton-like catalysis/immersed membrane separation system, J. Haz-ard. Mater. 244-245 (2013) 758–764.

ibitem{ref13} S. Karthikeyan, V.K. Gupta, R. Boopathy, A. Titus, G. Sekaran, A new approach for thedegradation of high concentration of aromatic amine by heterocatalytic Fenton ox-idation: kinetic and spectroscopic studies, J. Mol. Liq. 173 (2012) 153–163.

ibitem{ref14} Z. Weijiang, Z. Yace, G. Yuvaraja, X. Jiao, Adsorption of Pb(II) ions from aqueous en-vironment using eco-friendly chitosan Schiff’s base@Fe3O4 (CSB@Fe3O4) as an ad-sorbent; kinetics, isotherm and thermodynamic studies, Int. J. Biol. Macromol. 105(2017) 422–430.

ibitem{ref15} Y.G. Abou El-Reash, A.M. Abdelghany, A.A. Elrazak, Removal and separation of Cu(II)from aqueous solutions using nano-silver chitosan/polyacrylamide membranes, Int.J. Biol. Macromol. 86 (2016) 789–798.

ibitem{ref16} J. Chen, Y. Ma, L. Wang, W. Han, Y. Chai, T. Wang, J. Li, L. Ou, Preparation of chitosan/SiO2-loaded graphene composite beads for efficient removal of bilirubin, Carbon 143(2019) 352–361.

ibitem{ref17} K. Hamdi, P. Martin, M. Etienne, M. Hébrant, Rapid and reversible adsorption of BTXon mesoporous silica thin films for their real time spectrophotometric detection inair at ppm levels, Talanta 203 (2019) 269–273.

ibitem{ref18} [17] J. Chen, Y. Ma, L. Wang, W. Han, Y. Chai, T. Wang, J. Li, L. Ou, Preparation of chitosan/SiO2-loaded graphene composite beads for efficient removal of bilirubin, Carbon 143(2019) 352–361.

ibitem{ref19} [18] K. Hamdi, P. Martin, M. Etienne, M. Hébrant, Rapid and reversible adsorption of BTXon mesoporous silica thin films for their real time spectrophotometric detection inair at ppm levels, Talanta 203 (2019) 269–273.

ibitem{ref20} [19] S.K. Shukla, A.K. Mishra, O.A. Arotiba, B.B. Mamba, Chitosan-based nanomaterials: a state-of-the-art review, Int. J. Biol. Macromol. 59 (2013) 46–58. 767 S.K. Lakkaboyana, K. Soontarapa, Vinaykumar et al. International Journal of Biological Macromolecules 168 (2021) 760–768

ibitem{ref21} [39] H. Liang, B. Song, P. Peng, G. Jiao, X. Yan, D. She, Preparation of three-dimensionalhoneycomb carbon materials and their adsorption of Cr(VI), Chem. Eng. J. 367(2019) 9–16.

ibitem{ref22} [40] W. Sun, S.-L. Chen, M. Xu, Y. Wei, T.-t. Fan, J. Guo, The diffusion of molecules insideporous materials with bidisperse pore structures, Chem. Eng. J. 365 (2019) 201–219.

ibitem{ref23} [41] N.K. Asmel, A.R.M. Yusoff, L. Sivarama Krishna, Z.A. Majid, S. Salmiati, High concen-tration arsenic removal from aqueous solution using nano-iron ion enrich material(NIIEM) super adsorbent, Chem. Eng. J. 317 (2017) 343–355.

ibitem{ref24} [42] S.K. Lakkaboyana, S. Khantong, N.K. Asmel, A. Yuzir, W.Z. Wan Yaacob, Synthesis ofcopper oxide nanowires-activated carbon (AC@CuO-NWs) and applied for removalmethylene blue from aqueous solution: kinetics, isotherms, and thermodynamics, J.Inorg. Organomet. Polym. Mater. 29 (2019) 1658–1668.

ibitem{ref25} [43] H. Karaer, İ. Kaya, Synthesis, characterization of magnetic chitosan/active charcoalcomposite and using at the adsorption of methylene blue and reactive blue4, Micro-porous Mesoporous Mater. 232 (2016) 26–38.

ibitem{ref26} [44] L.S. Krishna, A. Yuzir, G. Yuvaraja, V. Ashokkumar, Removal of Acid Blue25 fromJ. aqueous solutions using Bengal gram fruit shell (BGFS) biomass,Phytoremediat. 19 (2017) 431–438. Int.

ibitem{ref27} [45] M.A.K.M. Hanafiah, W.S.W. Ngah, S.H. Zolkafly, L.C. Teong, Z.A.A. Majid, Acid Blue 25adsorption on base treated Shorea dasyphylla sawdust: kinetic, isotherm, thermo-dynamic and spectroscopic analysis, J. Environ. Sci. 24 (2012) 261–268.

ibitem{ref28} [46] Z.-X. Han, Z. Zhu, D.-D. Wu, J. Wu, Y.-R. Liu, Adsorption kinetics and thermodynam-ics of Acid Blue 25 and methylene blue dye solutions on natural sepiolite, Synthesisand Reactivity in Inorganic, Metal-organic, and Nano-metal Chemistry, 44, 2014,pp. 140–147.

ibitem{ref29} M.A.K.M. Hanafiah, W.S.W. Ngah, S.H. Zolkafly, L.C. Teong, Z.A.A. Majid, Acid Blue 25adsorption on base treated Shorea dasyphylla sawdust: kinetic, isotherm, thermo-dynamic and spectroscopic analysis, J. Environ. Sci. 24 (2012) 261–268.

ibitem{ref30} Z.-X. Han, Z. Zhu, D.-D. Wu, J. Wu, Y.-R. Liu, Adsorption kinetics and thermodynam-ics of Acid Blue 25 and methylene blue dye solutions on natural sepiolite, Synthesisand Reactivity in Inorganic, Metal-organic, and Nano-metal Chemistry, 44, 2014,pp. 140–147.

ibitem{ref31} L.S. Krishna, A.S. Reddy, W.Y.W. Zuhairi, M.R. Taha, A.V. Reddy, Indian jujuba seedpowder as an eco-friendly and a low-cost biosorbent for removal of acid blue 25from aqueous solution, Sci. World J. 2014 (2014), 184058, .

ibitem{ref32} R.S. Juang, R.L. Tseng, F.C. Wu, S.J. Lin, Use of chitin and chitosan in lobster shellwastes for color removal from aqueous solutions, J. Environ. Sci. Health, Part A: En-viron. Sci. Eng. Toxic Hazard. Subst. Control 31 (1996) 325–338.

ibitem{ref33} K. Badii, F.D. Ardejani, M.A. Saberi, N.Y. Limaee, Adsorption of acid blue 25 dye on diatomite in aqueous solutions, Indian J. Chem. Technol. 17 (2010) 7–16.

ibitem{ref34} F. Ferrero, Dye removal by low cost adsorbents: hazelnut shells in comparison with wood sawdust, J. Hazard. Mater. 142 (2007) 144–152.

ibitem{ref35} M.R.R. Kooh, M.K. Dahri, L.B.L. Lim, L.H. Lim, C.M. Chan, Separation of acid blue 25from aqueous solution using water lettuce and agro-wastes by batch adsorptionstudies, Appl Water Sci 8 (2018) 61.

ibitem{ref36} Y.S. Ho, G. McKay, Kinetic models for the sorption of dye from aqueous solution by wood, Process. Saf. Environ. Prot. 76 (1998) 183–191.

ibitem{ref37} L.S. Krishna, K. Soontarapa, N.K. Asmel, A. Yuzir, W.Y.W. Zuhairi, Effect ofcetyltrimethylammonium bromide on the biosorption of Acid Blue 25 onto Bengalgram fruit shell, 150 (2019) 386–395.

ibitem{ref38} L.S. Krishna, K. Soontarapa, N.K. Asmel, M.A. Kabir, A. Yuzir, W.Y.W. Zuhairi, Y. Sarala,Adsorption of acid blue 25 from aqueous solution using zeolite and surfactant mod-ified zeolite, Desalin. Water Treat. 150 (2019) 348–360.

ibitem{ref39} [53] L.S. Krishna, K. Soontarapa, N.K. Asmel, A. Yuzir, W.Y.W. Zuhairi, Effect ofcetyltrimethylammonium bromide on the biosorption of Acid Blue 25 onto Bengalgram fruit shell, 150 (2019) 386–395.

ibitem{ref40} [54] L.S. Krishna, K. Soontarapa, N.K. Asmel, M.A. Kabir, A. Yuzir, W.Y.W. Zuhairi, Y. Sarala,Adsorption of acid blue 25 from aqueous solution using zeolite and surfactant mod-ified zeolite, Desalin. Water Treat. 150 (2019) 348–360.

ibitem{ref41} [55] S.K. Lakkaboyana, S. Khantong, M.A. Kabir, Y. Ali, W.Z.W. Yaacob, Removal of AcidBlue 25 dye from wastewater using Rambutan (Nephelum lappaceum Linn) seed asan efficient natural biosorbent, 6 (2018) 111–117.

ibitem{ref42} [20] M. Dash, F. Chiellini, R.M. Ottenbrite, E. Chiellini, Chitosan-A versatile semi-synthetic polymer in biomedical applications, Prog. Polym. Sci. 36 (2011) 981–1014.

ibitem{ref43} [21] L. Zhai, Z. Bai, Y. Zhu, B. Wang, W. Luo, Fabrication of chitosan microspheres for ef-ficient adsorption of methyl orange, Chin. J. Chem. Eng. 26 (2018) 657–666.

ibitem{ref44} [22] H. Zhao, J. Xu, W. Lan, T. Wang, G. Luo, Microfluidic production of porous chitosan/silica hybrid microspheres and its Cu(II) adsorption performance, Chem. Eng. J. 229(2013) 82–89.

ibitem{ref45} [23] E. Igberase, P. Osifo, Equilibrium, kinetic, thermodynamic and desorption studies ofcadmium and lead by polyaniline grafted cross-linked chitosan beads from aqueoussolution, J. Ind. Eng. Chem. 26 (2015) 340–347.

ibitem{ref46} [24] R.K. Marella, V.R. Madduluri, S.K. Lakkaboyana, M.M. Hanafiah, S. Yaaratha,Hydrogen-free hydrogenation of nitrobenzene via direct coupling withcyclohexanol dehydrogenation over ordered mesoporous MgO/SBA-15 supportedCu nanoparticles, RSC Adv. 10 (2020) 38755–38766.

ibitem{ref47} [25] G.M. Raghavendra, J. Jung, D. kim, J. Seo, Microwave assisted antibacterial chitosan– silver nanocomposite films, Int. J. Biol. Macromol. 84 (2016) 281–288.

ibitem{ref48} [26] K.D. Khalil, S.M. Riyadh, S.M. Gomha, I. Ali, Synthesis, characterization and applica-tion of copper oxide chitosan nanocomposite for green regioselective synthesis of[1,2,3]triazoles, Int. J. Biol. Macromol. 130 (2019) 928–937.

ibitem{ref49} [27] Q. Li, J. Zhou, L. Zhang, Structure and properties of the nanocomposite films of chi-tosan reinforced with cellulose whiskers, J. Polym. Sci. B Polym. Phys. 47 (2009)1069–1077.

ibitem{ref50} K.D. Khalil, S.M. Riyadh, S.M. Gomha, I. Ali, Synthesis, characterization and applica-tion of copper oxide chitosan nanocomposite for green regioselective synthesis of[1,2,3]triazoles, Int. J. Biol. Macromol. 130 (2019) 928–937.

ibitem{ref51} Q. Li, J. Zhou, L. Zhang, Structure and properties of the nanocomposite films of chi-tosan reinforced with cellulose whiskers, J. Polym. Sci. B Polym. Phys. 47 (2009)1069–1077.

ibitem{ref52} A. Temiz, N. Terziev, M. Eikenes, J. Hafren, Effect of accelerated weathering on sur- face chemistry of modified wood, Appl. Surf. Sci. 253 (2007) 5355–5362.

ibitem{ref53} M. Yamada, I. Honma, Anhydrous proton conductive membrane consisting of chito- san, Electrochim. Acta 50 (2005) 2837–2841.

ibitem{ref54} L. Lu, H. Sun, F. Peng, Z. Jiang, Novel graphite-filled PVA/CS hybrid membrane forpervaporation of benzene/cyclohexane mixtures, J. Membr. Sci. 281 (2006)245–252.

ibitem{ref55} M.W. Sabaa, H.M. Abdallah, N.A. Mohamed, R.R. Mohamed, Synthesis, characteriza-tion and application of biodegradable crosslinked carboxymethyl chitosan/poly(vinyl alcohol) clay nanocomposites, Mater. Sci. Eng. C 56 (2015) 363–373.

ibitem{ref56} M. Monier, D.M. Ayad, Y. Wei, A.A. Sarhan, Preparation and characterization of mag-netic chelating resin based on chitosan for adsorption of Cu(II), Co(II), and Ni(II)ions, React. Funct. Polym. 70 (2010) 257–266.

ibitem{ref57} I.P. Merlusca, D.S. Matiut, G. Lisa, M. Silion, L. Gradinaru, S. Oprea, I.M. Popa, Prepa-ration and characterization of chitosan–poly(vinyl alcohol)–neomycin sulfate films,Polym. Bull. 75 (2018) 3971–3986.

ibitem{ref58} A. Rahnama, M. Gharagozlou, Preparation and properties of semiconductor CuOnanoparticles via a simple precipitation method at different reaction temperatures,Opt. Quant. Electron. 44 (2012) 313–322.

ibitem{ref59} L. You, C. Huang, F. Lu, A. Wang, X. Liu, Q. Zhang, Facile synthesis of high perfor-mance porous magnetic chitosan - polyethylenimine polymer composite forCongo red removal, Int. J. Biol. Macromol. 107 (2018) 1620–1628.

ibitem{ref60} U.-J. Kim, S. Kimura, M. Wada, Highly enhanced adsorption of Congo red ontodialdehyde cellulose-crosslinked cellulose-chitosan foam, Carbohydr. Polym. 214(2019) 294–302.

ibitem{ref61} A.B. Albadarin, M.N. Collins, M. Naushad, S. Shirazian, G. Walker, C. Mangwandi, Ac-tivated lignin-chitosan extruded blends for efficient adsorption of methylene blue,Chem. Eng. J. 307 (2017) 264–272.

ibitem{ref62} A. Kamari, W.S.W. Ngah, M.Y. Chong, M.L. Cheah, Sorption of acid dyes onto GLA and H2SO4 cross-linked chitosan beads, Desalination 249 (2009) 1180–1189. 768

\end{{thebibliography}}
\end{document}
